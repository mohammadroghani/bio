\documentclass[11pt, oneside]{article}   	
\usepackage[hidelinks]{hyperref}
\usepackage{cite}
\usepackage{url}
\usepackage{xepersian}
\settextfont{XB Zar}
\title{گزارش تئوری}
\author{محمد روغنی}
\begin{document}
\maketitle
\section*{1.2}
\begin{itemize}
\item پروتئین های ساختاری
\cite{stru}
\begin{enumerate}
\item VP24
\item VP30
\item VP35
\item VP30
\item GP
\item NP
\item L

\end{enumerate}
\item پروتئین های غیرساختاری
\begin{enumerate}
\item sGP
\end{enumerate}
\end{itemize}
پروتئین VP24 و VP40 در انتشار و بازتولید ویروس ابولا نقش مهمی دارند.
\section*{1.3}
ابولا از طریق دنریت‌ها که رشته مانندی هستند که به جسم سلولی یاخته‌های عصبی (نورون‌ها)، متصل هستند به غدد لنفاوی نفوز کرده و سپس با وارد شدن به خون و رسیدن به کبد باعث بیماری و کشته شدن فرد می‌شود.
\cite{ebola}

\section*{3.1}
\begin{itemize}
\item در الگوریتم UPGMA ارتفاع برگ ها برابر است اما در الگوریتم NJ متفاوت است.
\item درخت خروجی از الگوریتم UPGMA دارای ریشه است اما درخت خروجی از الگوریتم NJ بدون ریشه است.
\end{itemize}
این تفاوت ها به دلیل تفاوت در نوع الگوریتم هاست، در الگوریتم UPGMA، molecule clock ثابت در نظر گرفته میشود ولی در الگوریتم NJ خیر.\\
در مسائلی که molecular clock ثابت است میتوان از UPGMA استفاده کرد اما الگوریتم دقیقی نیست.
\cite{upgmanj}
\section*{3.2}
\begin{itemize}
\item روش اول
در این روش میتوانیم ابتدا بیابیم هر ژن به چه قسمتی از یک ژنوم منطبق میشود. مثلا برای ژنوم سودان محل انطباق تمام ۷ ژن را به دست آوریم. سپس این قسمت ها را به هم بچسبانیم و بعد روی ژن های جدید الگوریتم تطابق سراسری را اجرا کنیم. این روش بسیار دقیق است چرا که قسمت های منطبق شده با ژن قسمت های تاثیرگذار هستند و جهش در آن ها بیشترین تاثیر در تفاوت عملکرد گونه دارد.
\item روش دوم
در این روش از کتابخانه ی  ape در R تابع consensus را فراخوانده و ۷ ماتریس فاصله در قسمت قبل را به آن میدهیم، خروجی درخت تلفیقی از این ۷ درخت خواهد بود.
\end{itemize}
\section*{4.1}
$$rt = -ln(1- \frac{4}{3}p)$$
\begin{itemize}
\item p = نسبت تعداد جهش ها به طول رشته ی اولیه
\item r = نرخ جهش
\end{itemize}
این مدل بر پایه ی مدل jukes cantor است که در آن احتمال جهش ها برابر در نظر گرفته میشود.
\cite{time}
\section*{4.2}
میتوانیم از این مدل استفاده کنیم که اگر تعداد جهش ها از مقداری بیشتر شد بگوییم یک گونه ی جدید به وجود آمده است. باتوجه به فاصله ی ویرایش گونه های مختلف ویروس ابولا میتوانیم این حد را برابر با میانگین فاصله ی ویرایش دو به دوی آن ها قرار دهیم که برابر است با ۶۵۴۳، و طول میانگین این گونه ها برابر است با ۱۹۰۰۰ که طبق فرمول بخش قبل زمان لازم برای به وجود آمدن یک گونه ی جدید برابر خواهد بود با ۳۲۳ سال.
\newpage
\begin{thebibliography}{1}
\begin{latin}
\bibitem{upgmanj}
   {\url{https://www.researchgate.net/post/What_is_the_difference_between_UPGMA_and_NEJ_method_while_constructing_a_tree_using_a_MEGA_4_software}}
\bibitem{time}
   {\url{https://mathcs.clarku.edu/~djoyce/java/Phyltree/mutations.html}}  
\bibitem{stru}
   {\url{https://en.wikipedia.org/wiki/Ebola_virus}}  
\bibitem{ebola}
   {\url{http://www.sciencemag.org/news/2014/08/what-does-ebola-actually-do}}  

\end{latin}
\end{thebibliography}
\end{document}  